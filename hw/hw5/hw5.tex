\documentclass[12pt, letterpaper]{article}

\usepackage[utf8]{inputenc}
\usepackage{mathtools}
\usepackage[left=0.6in, right=0.6in, top=1in, bottom=1in]{geometry}
% \usepackage{graphicx}
% \usepackage{algorithm}
% \usepackage{algpseudocode}
% \usepackage{changepage}
% \usepackage{xcolor}
% \usepackage{amsmath}

% for code snippets
\usepackage{listings}
\usepackage{color}

\definecolor{dkgreen}{rgb}{0.5,0.65,0.5}
\definecolor{gray}{rgb}{0.5,0.5,0.5}
\definecolor{code_blue}{rgb}{0.10,0.27,0.92}
\definecolor{str_red}{rgb}{0.87,0.2,0.14}

\lstset{frame=tb,
  language=Java,
  aboveskip=3mm,
  belowskip=3mm,
  showstringspaces=false,
  columns=flexible,
  basicstyle={\small\ttfamily},
  numbers=none,
  numberstyle=\tiny\color{gray},
  keywordstyle=\color{code_blue},
  commentstyle=\color{dkgreen},
  stringstyle=\color{str_red},
  breaklines=true,
  breakatwhitespace=true,
  tabsize=3
}

\title{CSE 101 Homework 5}
\author{Brian Masse, Kreshiv Chawla, Taira Sakamoto, Emily Xie, Annabelle Coles}
\date{October 31, 2025}

\begin{document}

\maketitle
\newpage

\begin{enumerate}

% MARK: Question 1
\newpage
\item \textbf{Cart horses}

Consider the following problem: We have  $n$ horses, $Horse_1,..Horse_n$, each with a cart-pulling
speed $s_i$.  We need to pair the horses up into teams to pull carts.The faster horse will need to slow down for the slower horse,so if we pair $Horse_i$ and $Horse_j$, the cart will be pulled at speed $min(s_i,s_j)$.  Each horse can only be in one pair. We want to maximize the sum of all the cart speeds. 

Give a greedy strategy for this problem, and prove that it is correct using any of the proof templates from class.


\begin{enumerate}
    \item Greedy Algorithm: When picking the next pair of horses, pick the 2 fastest horses from the remaining horses.
    \item Proof of correctness (via exchange):
    
    Let $G$ be the first greedy choice. Let $OS$ be an optimal solution

    Find the 2 pairs with the 2 fastest horses, $H_1, H_2$: $(H_1, H_i), (H_2, H_j)$. These horses will be separate, otherwise $OS$ already contains $G$. Exchange $H_i$ and $H_2$

    Note:
    \begin{eqnarray*}
        && H_1 \ge H_2, H_2 \ge H_i, H_2 \ge H_j \\
        && min(H_1, H_i) = H_i, min(H_2, H_j) = H_j \\
        &\implies& min(H_1, H_2) \ge min(H_1, H_i)
    \end{eqnarray*}
    
    For the second new pairing, $(H_i, H_j)$ there are 2 possibilities
    \begin{enumerate}
        \item $H_i \ge H_j \implies min(H_2, H_j) = min(H_i, H_j) = H_j$:
        \item $H_i < H_j:$ 
        \begin{eqnarray*}
            && min(H_1, H_i) + min(H_2, H_j) - min(H_1, H_2) - min(H_i, H_j) \\
            &=& H_i + H_j - H_2 - H_i = H_j - H_2 \le 0 \\
            &\implies& \textnormal{New solution is faster}
        \end{eqnarray*}
    \end{enumerate}

    Thus, the net speed of the exchanged solution is always as good as the original and meets the constraints of the problem

    \-\ \newline
    Now, proving correctness via induction on number of horses. This proof assumes there will always be an even number of horses to make the pairings.

    \begin{enumerate}
        \item Base Case: $n = 0, 2$: Trivially, the greedy solution (and all solutions) are optimal
        \item Induction Step. Let $OS$ be any solution for the group of horses $I = \{H_1,...H_k\}, k > 2$
        
        By the exchange argument above, $\exists OS'$ such that $|OS| \le |OS'|$ and $G \in OS$

        Let $I'$ be the set of pairs of horses that have neither $H_1$ or $H_2$ in them. 

        \begin{eqnarray*}
            && OS' = \{G\} \cup I' = \{ (H_1, H_2), (H_i, H_j) \} \cup I' \\
            && \textnormal{Because } ||I'|| = k - 4 \le k, \textnormal{by the induction hypothesis } |S(I')| \le |GS(I')| \\
            &\implies& |OS| \le |OS'| = |\{G\} \cup OS(I')| \le |\{G\} \cup GS(I')| = |GS(I)|
        \end{eqnarray*}
    \end{enumerate}

    Thus, for every set of horses, the greedy solution is as good as the optimal solution. 


    \item Time Analysis.
     This algorithm only needs an input of horses, sorted by their speed. Thus, the algorithm takes $O(n \cdot log(n))$
     \begin{itemize}
        \item $log(n)$ to sort the horses
        \item $n$ to scan through the sorted list of horses and create the pairings
     \end{itemize}


\end{enumerate}



% MARK: Question 2
\newpage
\item \textbf{Sensor maximization}

Suppose you are placing sensors on a one-dimensional road.  You have identified $n$ possible locations for sensors, at distances $0 \le d_1\le d_2 \le ...\le d_n$ from the start of the road. Each sensor must be at most $M$ distance from the previous one (so they can communicate reliably). The first one must be at $0$ and the last at $d_n$. Given that, you want to minimize the number of sensors placed.  

The following greedy algorithm, which places each sensor as far as possible from the previous one, will return a list $d_{i_1} \le d_{i_2} \le ...d_{i_k}$ of locations where sensors can be placed.  

GreedySensorMin [$d_1...d_n, M$] 
\begin{enumerate}
\item Initialize a  list to ($0$).
\item Initialize $I=1$, $PreviousSensor=0$.
\item While $I \le n$ do:
\item ~~~~~~While $I \le n$ and $d_{I} < PreviousSensor+ M$ do:$I++$.
\item ~~~~~~If $I \le n$ THEN append $d_{I-1}$ to list; $PreviousSensor=d_I$;$I++$. 
\item Append $d_n$ to list.
\item Return list
\end{enumerate}  

Questions:

\begin{enumerate}
\item What constraints do solutions $d_{i_1},..d_{i_k}$ need to meet for this problem?

\begin{itemize}
    \item For a solution $d_{i_1},...d_{i_k}$, minimize k
    \item $d_{i_1} < d_{i_2} < \dots < d_{i_k}$
    \item $d_{i_j} - d_{i_{j-1}} < M \implies d_{i_j} < d_{i_{j - 1}} + M$
    \item ${d_{i_1}} = 0, d_{i_k} = d_n$
\end{itemize}

\-\ 
\item What is the value of the objective function for a solution, of the form $d_0,d_{i_1},..d_{i_k}=d_n$, ?  
The value of the objective function of such a solution is $k$. The goal is to minimize k 

\-\
\item In using the ``greedy stays ahead'' proof technique to show that this is optimal, we would compare the greedy solution $d_{g_1},..d_{g_k}$ to another solution, $d_{j_1},..d_{j_{k'}}$ .   In what sense is the greedy solution ``staying ahead'' of the other solution at each step $1 \le t \le  k'$?

The $i^{th}$ choice to place a marker in the greed solution must always be farther or equally along the road compared to the $i^{th}$ choice in the normal solution. ($d_{g_i} \ge d_{j_i}$)

\-\
\item Prove the claim you wrote above using induction on the step $t$.  

\begin{enumerate}
    \item Base case: $t = 0 \implies d_{g_0} = d_0 = 0 = d_{j_0}$ 
    \item Induction Step: Assume $d_{g_{t - 1}} \ge d_{j_{t-1}}$, show $d_{g_{t}} \ge d_{j_{t}}$ for $t > 0$
    
    \begin{itemize}
        \item Let I be the set of valid choices for the greedy algorithm at step $t$. It contains markers between $d_{g_{t-1}}, d_{g_{t-1}} + M$
        \item Let J be the set of valid choices for the regular optimal solution. It contains markers between $d_{j_{t-1}}, d_{j_{t-1}} + M$
    \end{itemize}

    By the inductive hypothesis, $d_{g_{t-1}} \ge d_{j_{t-1}} \implies d_{g_{t-1}} + M \ge d_{j_{t-1}} + M$. Thus, any valid solution in the regular set, $j \in J$, is either less than all solutions in I, or contained within I. Because the Greedy algorithm picks the largest marker in I: 
    \begin{itemize}
        \item If the regular solution picks $d_{j_t} \notin I \implies d_{j_t} < d_{g_t}$
        \item If the regular solution picks $d_{j_t} \in I \implies d_{j_t} \le d_{g_t}$
    \end{itemize}

    Thus, the $i^{th}$ greedy choice, $d_{g_i}$, is at least as far along the as the $i^{th}$ normal choice $d_{j_i}$
\end{enumerate}

\-\
\item In O notation, how much time does the algorithm as written take?

The above algorithm runs in $O(n)$ time. Each one of the $n$ markers is considered ($O(1)$) exactly once.
\end{enumerate}





% MARK: Question 3
\newpage
\item \textbf{Quests}

In a role-playing game, your character will complete a series of $k$ quests.  There is a list of $n> k$ possible quests $Q_i$,  each with a first time it can be attempted $k \ge f_i \ge 1$ and the amount of gold earned on the quest, $g_i>0$. You cannot complete the same quest twice, and you cannot put quest $i$ in a position before $f_i$ in your list. You want to maximize your total gold at the end of all your quests. 

\begin{enumerate}
\item Clearly state a greedy strategy that gives the most possible total gold

At choice $i$ pick the quest with the most gold and that satisfies $f_i \le i$


\item Prove that this strategy is correct. Hint: use a modify-the-solution proof with two cases, based on whether the next quest the greedy algorithm performs is performed at some point in $OS$.

\-\ \newline
Proof by exchange:
\begin{itemize}
    \item Let $G$ be the first greedy solution ($Q{g_o}$)
    \item Let $OS$ be an optimal solution
\end{itemize}

There are 2 possibilities:
\begin{enumerate}
    \item $OS$ never picks $G$:
    
    In this case, create $OS'$ by exchanging the first non greed choice, $Q_{j_0}$ with G
    \begin{itemize}
        \item $OS'$, by definition, still satisfies the ordering constraint ($f_{g_0}$ must $=1$ to be chosen first)
        \item $G$ yields the highest gold at choice 1: $g_{g_0} \ge g_{j_0} \implies |OS'| \ge |OS|$
    \end{itemize}

    \-\
    \item $OS$ Picks $G$ as the $i^{th}$ quest.
    
    In this case, exchange $G$ with the first pick, $Q_{j_0}$
    \begin{itemize}
        \item Still satisfies constraints: both $f_{j_0}$ and $f_{g_0} \le 1$. Thus both can serve as the first and $i^{th}$ choice without breaking a constraint
        \item $|OS'| = |OS|$ since all the quests are the same
    \end{itemize}
\end{enumerate}

Thus, you can always create a $|OS'|$ from an $OS$, such that $OS'$ meets the constraints of the problem and is equally as good.

\-\ \newline
Now, proving correctness via induction on number of quests. 
\begin{enumerate}
    \item Base case: n = 1: Greedy (and all) solutions are trivially correct
    \item Induction Step:
    
    \begin{itemize}
        \item Let $I$ be a set of quests $\{Q_1,...Q_n\}$. We can create $OS'$ from any $OS(I)$ such that $|OS'| \ge |OS|$ and $G \in OS'$
        \item Let $I'$ be the set of quests without $G$
        \item Let $I''$ be the set of quests where $G$ is not the first choice. 
    \end{itemize}

    Note that by the above proof, we can trivially exchange any solution $OS(I'')$ into a solution on $I$ starting with $G$ with changing the optimality of the solution. Thus,
    \begin{eqnarray*}
        OS' &=& \{G\} \cup I' \cup I'' \\
        |OS| \le |OS'| &=& |\{G\} \cup OS(I') \cup OS(I'')| \\
        &=& |\{G\} \cup OS(I')| \textnormal{ (By exchanging solutions from $I''$)} \\
        &\le& |\{G\} \cup GS(I')| \textnormal{ (By Induction Hypothesis)} \\
        &=& |GS(I)|
    \end{eqnarray*}
\end{enumerate}

Thus, the greedy solution provides an optimal solution on any set of quests. 




\-\ 
\item Give an efficient algorithm carrying out this strategy and give a time analysis of your algorithm.
\end{enumerate}


% MARK: Question 4
\newpage
\item \textbf{Minimal spanning subgraph with negative edge weights}

Say we allow negative edge weights into the spanning tree problem. We are given a connected undirected graph $G=(V,E)$ with possibly negative edge weights. We wish to find the set $E' \subset E$ so that $G'=(V,E')$ is connected that minimizes $\sum_{e \in E'} w(e)$.

\begin{enumerate}
\item Give an example of a graph with some negative edge weights where the minimum cost connected subgraph is not a tree.
\item Which edges are always in the minimum cost subgraph?
\item Once the edges above are added, what is an equivalent problem?
\item Describe how to use the answers to the above two questions to reduce this problem to minimum spanning tree. Give a time analysis for the resulting algorithm.
\end{enumerate}

% MARK: Question 5
\newpage
\item \textbf{Graph coloring heuristics}

The graph coloring problem is, given an undirected graph, give each vertex a color, so that neighboring vertices have different colors.  You want to minimize the number of colors.  This is an $NP$-complete problem, so we might want to use a greedy heuristic.

One greedy heuristic would be to order the vertices, then assign each the smallest color that is different from its neighbors.   

Implement this heuristic on random graphs where we start with $|V|$ vertices and add $5|V|$ random edges. For a variety of sizes (e.g., $|V|=2^k$ for $k=3,...12$), and several graphs of each size, try the heuristic ordering the vertices by increasing degree (number of neighbors) and decreasing degree. Plot the two numbers of colors used. Do these increase with $|V|$?  How consistent are the results with different graphs of the same size?  


\end{enumerate}
\end{document}

