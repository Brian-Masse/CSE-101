\documentclass[12pt, letterpaper]{article}

\usepackage[utf8]{inputenc}
\usepackage{mathtools}
\usepackage[a4paper, total={6in, 9in}]{geometry}

\title{CSE 101 Homework 1}
\author{Brian Masse, Emily Xie, Kreshiv Chawla}
\date{October 7, 2025}

\begin{document}

\maketitle
\newpage

% MARK: Problem 1a
% problem statement
\bf{ 1. Following the Algorithm Defined in HW1. Consider the case when \(A[i] + A[j] = V\) for some \(1 \le i < j \le n\). Prove the invariant: If the while loop has not terminated, then \(I \le i < j \le J \) }

% proving the base case
\-\ \newline
\-\ \it{
    i) Prove the base case: (I = 1, J = n, assuming n \(>\) 1.) 
}
\-\ \newline
\-\ \newline
\textnormal{
    Given \(1 \le i < j \le n \implies I \le i < j \le J \)
}

\-\ \newline
\-\ \textnormal{
    ii) Prove the general case:
}
\-\ \newline
\-\ \newline
\textnormal{
    Assume that after \(x > 1\) iterations, \(I \le i < j \le J\). Show the loop invariant after the \(x + 1\) iteration.
}
\-\ \newline
\-\ \newline
If \(i = I\) and \(j = J\) \(\implies\)

\-\ \-\ \(A[i] + A[j] = V \implies\) \textnormal{returns True}
\-\ \newline
If \(j = J\)
\-\
\begin{flalign}
&\implies I < i\\
&\implies \textnormal{(By Sorted Array)
 \( A[I] + A[J] = A[I] + A[j] < V\)}\\
&\implies I\textnormal{++}\\
&\implies I \le i < j \le J
\end{flalign}

\-\ \newline
If \(i = I\)
\-\
\begin{flalign}
&\implies J > j\\
&\implies \textnormal{(By Sorted Array)
 \( A[I] + A[J] = A[i] + A[J] > V\)}\\
&\implies J\textnormal{- -}\\
&\implies I \le i < j \le J
\end{flalign}

\-\ \newline
Else:
\-\
\begin{flalign}
&\implies J > j \textnormal{ \& } i < I\\
&\implies \textnormal{ Either I++ or J-- will uphold the loop invariant.}
\end{flalign}

\end{document}