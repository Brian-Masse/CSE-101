\documentclass[12pt, letterpaper]{article}

\usepackage[utf8]{inputenc}
\usepackage{mathtools}
\usepackage[a4paper, total={6in, 9in}]{geometry}

\title{CSE 101 Homework 1}
\author{Kreshiv Chawla, Brian Masse, Taira Sakamoto, Emily Xie}
\date{October 7, 2025}

\begin{document}

\maketitle
\newpage

\begin{enumerate}

%--------------------- Question 1 ---------------------
\item
\begin{enumerate}
  \item 
  \[
    \frac{2^{n/2}}{2^n} = 2^{-n/2} \xrightarrow[n \to \infty]{} 0.
  \]
  Hence, for any constant $c > 0$ and all sufficiently large $n$, 
  $2^{n/2} < c \, 2^n$. \textbf{False.}

  \item
  \[
    (n^3 + 2n + 1)^4 
      = n^{12} \! \left(1 + 2n^{-2} + n^{-3}\right)^4
      = n^{12} \! \left(1 + O(n^{-2})\right),
  \]
  \[
    (n^4 + 4n^2)^3
      = n^{12} \! \left(1 + 4n^{-2}\right)^3
      = n^{12} \! \left(1 + O(n^{-2})\right).
  \]
  Therefore,
  \[
    \frac{(n^3 + 2n + 1)^4}{(n^4 + 4n^2)^3}
      \xrightarrow[n \to \infty]{} 1 \neq 0.
  \]
  Not little-$o$. \textbf{False.}

  \item
  \[
    \log\!\left(n^{10}\right) - \log(\log n)
      = 10 \log n - \log \log n.
  \]
  Hence $\Theta(\log n)$. \textbf{True.}

  \item
  \[
    \sum_{i=1}^{n} i^{3}
      = \left(\frac{n(n+1)}{2}\right)^{2}
      = \frac{n^2 (n+1)^2}{4}
      = \frac{n^4 + 2n^3 + n^2}{4}.
  \]
  \textbf{True.}

  \item
  \[
    n! = (n)(n-1)(n-2)\cdots 2 \cdot 1
      \le n \cdot n \cdot n \cdots n = n^n.
  \]
  \textbf{True.}
\end{enumerate}

%--------------------- Question 2 ---------------------
\item
\textbf{Base case.} For $n = 2$,
\[
  F_2 = F_{2-1} + F_{2-2} = F_1 + F_0 = 1 + 0 = 1.
\]

\textbf{Inductive hypothesis.} Assume that the statement about $F_n$
holds for all integers up to $k$.

\textbf{Inductive step.}
\[
  F_{k+1}
    = F_k + F_{k-1}
    = (F_{k-1} + F_{k-2}) + F_{k-1}
    = F_k + F_{k-1},
\]
establishing the step.

%--------------------- Question 3 ---------------------
\item
\textbf{Base case.} Initially, there are $101$ black marbles (odd). \\[4pt]
\textbf{Inductive hypothesis.} Assume that after $k$ moves, the number of black marbles has the same parity as it had initially (odd). \\[4pt]
\textbf{Inductive step.} On move $k{+}1$, one of three things happens:
\begin{itemize}
  \item BB: black count changes by $-2$ (parity unchanged).
  \item RR: black count changes by $0$ (parity unchanged).
  \item BR: black count changes by $0$ (parity unchanged).
\end{itemize}
In every case, the parity of the black count after move $k+1$ equals the parity after move $k$.  
By the inductive hypothesis, this equals the initial parity.  
Hence, the claim holds for $k+1$. \qed

%--------------------- Question 4 ---------------------
\item
Following the Algorithm Defined in HW1.  
Consider the case when \( A[i] + A[j] = V \) for some \( 1 \le i < j \le n \).  
Prove the invariant: If the while loop has not terminated, then \( I \le i < j \le J \).

\begin{enumerate}
  \item \textbf{Base case:} \( I = 1, J = n \) (assuming \( n > 1 \)).  
  \[
    1 \le i < j \le n \implies I \le i < j \le J.
  \]

  \item \textbf{General case:}  
  Assume that after \( x > 1 \) iterations, \( I \le i < j \le J \).  
  Show the loop invariant after the \( x + 1 \)-th iteration.

  \begin{description}
    \item[Case 1:] \( i = I \) and \( j = J \)
    \[
      A[i] + A[j] = V \implies \textnormal{returns True.}
    \]

    \item[Case 2:] \( j = J \)
    \begin{flalign}
      & I < i, \\
      & \textnormal{By sorted array: } A[I] + A[J] = A[I] + A[j] < V, \\
      & I \textnormal{++}, \\
      & I \le i < j \le J.
    \end{flalign}

    \item[Case 3:] \( i = I \)
    \begin{flalign}
      & J > j, \\
      & \textnormal{By sorted array: } A[I] + A[J] = A[i] + A[J] > V, \\
      & J \textnormal{--}, \\
      & I \le i < j \le J.
    \end{flalign}

    \item[Case 4:] Else
    \begin{flalign}
      & J > j \textnormal{ and } i < I, \\
      & \textnormal{Either } I++ \textnormal{ or } J-- \textnormal{ will uphold the loop invariant.}
    \end{flalign}
  \end{description}
\end{enumerate}

\end{enumerate}

\end{document}
